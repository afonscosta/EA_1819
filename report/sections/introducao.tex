
O projeto \textbf{Home4All} é constituído por um \textit{website} de imóveis
onde os seus utilizadores podem consultar imóveis, ou mesmo comprar e arrendar.
Este é implementado como sendo uma aplicação \textit{web} e, como tal, pode ser
usado em qualquer dispositivo.

De facto, para atingir este objetivo são usadas diversas ferramentas
tecnológicas, que se integram de forma a oferecer um sistema coeso. Desta forma,
a linguagem de base escolhida foi o Java e, por isso, são utilizadas
\textit{frameworks} como o \texttt{Hibernate}, o \texttt{JEE} e o \texttt{Vue.js}, para a camada de dados, aplicacional e de
apresentação, respetivamente. 

Primeiramente, é descrita uma contextualização do projeto, onde se pretende dar
resposta ao porquê da necessidade de uma ferramenta como esta nos dias de hoje.
De seguida, descreve-se o domínio do problema e quais os constituintes deste.
Posteriormente, tem-se uma análise detalhada de todos os requisitos inerentes ao
projeto, quer ao nível do utilizador como do sistema. A próxima etapa envolve a
conceção de um \textit{mockup} da interface, assim como a primeira fase de
testes de usabilidade. Além disso, tem-se o desenvolvimento propriamente dito,
com especial foco nos problemas encontrados e possíveis melhorias em iterações
futuras do projeto. São ainda apresentados os resultados de testes de carga efetuados à aplicação desenvolvida, de forma a avaliar o seu desempenho em situações de \textit{stress}. Por fim, tem-se uma nova fase de avaliação da usabilidade,
onde o objetivo é concluir os aspetos positivos e negativos do projeto.
