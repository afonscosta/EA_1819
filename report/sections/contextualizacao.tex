% Problema e Motivação

O mercado imobiliário digital já existe à algum tempo e tem vindo a tornar-se
a norma no que toca à forma como as pessoas adquirem/arrendam imóveis.
Efetivamente, à uns anos atrás o normal seria que as pessoas se dirigissem a
uma imobiliária física para perceber as suas possibilidades. No entanto, hoje
em dia o primeiro instinto é, normalmente, usar os meios digitais para
realizar a pesquisa. De facto, não é à toa que assim o é, pois tem-se uma
maior rapidez nas pesquisa realizadas, tem-se acesso instantâneo a imagens e
dimensões dos imóveis, assim como um possível contacto para interagir com o
anunciante.

Contudo, apesar do mercado estar bastante explorado, existe alguns aspetos que
podem ser melhorados. Nomeadamente, no que toca ao arrendamento de
apartamentos/quartos, rapidamente se percebe que existe uma quantidade
substancial de informação dispersa ou, em vários \textit{sites} especializados, 
redes sociais ou mesmo através dos \textit{emails}. 

Mais ainda, existe normalmente muitas possibilidades que correspondem ao que o
cliente pretende, tornando-se complicado para o utilizador cobrir essa
panóplia de opções. Além disso, a maior parte dos sistemas apenas funcionam a
partir de interações diretas com o utilizador, ou seja, através de pesquisas,
comparações ou mesmo adicionar aos favoritos. No entanto, achou-se necessário
ser o próprio sistema a fazer algum desse trabalho e proativamente tomar a
iniciativa de interagir com o utilizador, como por exemplo através de
notificações. Mais ainda, dado que a compra de um imóvel é sempre uma decisão
de peso, achou-se que os serviços existentes não oferecem as funcionalidades
necessárias para estreitar a relação existente entre o cliente e o anunciante,
de forma a dar uma maior segurança na altura de realizar o investimento.

O projeto \textbf{Home4All} surge como um projeto ambicioso, que tem como objetivo 
dar resposta a alguns do problemas mencionados anteriormente. Para tal, é
importante que este incorpore um sistema de caracterização de imóveis bastante
expressivo, de forma a permitir uma maior especificidade na pesquisa por parte
do cliente; é importante que permita configurar notificações para imóveis que
o cliente esteja interessado; que permita interagir de forma direta com o
anunciante; que permita anunciar imóveis de forma simples e rápida e que tenha
um \textit{design} renovado e com elevado grau de usabilidade.

Assim sendo, este documento tem como objetivo documentar todo o processo que
leva o Home4All a implementar algumas destas funcionalidades. De realçar que o
objetivo passa sempre por melhorar a forma como os utilizadores realizam compras 
de imóveis \textit{online}.
