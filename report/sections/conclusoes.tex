

Com este trabalho foi possível a perceção da importância da fase de conceção, no desenvolvimento do \textit{software}. De facto, a análise dos requisitos do sistema tendo em conta os utilizadores e o estabelecimento de \textit{personas} permitiu-nos compreender melhor o que as entidades que vão utilizar a aplicação pretendem da mesma e, desta forma, desenvolver um sistema mais focado nos mesmos. A realização dos testes de usabilidade, por sua vez, permitiram perceber que se estes tivessem sido efetuados atempadamente, teriam permitido a melhoria dos \textit{mockups} desenvolvidos e, consequentemente, da interface final. 

Relativamente à fase de desenvolvimento da aplicação, verificou-se, mais uma vez, a utilidade de diagramas como os de classe. Neste projeto, o recurso a diagramas dependentes de tecnologias (PSM), permitiu uma inicialização do desenvolvimento bastante mais rápida. Para além disso, estes diagramas facilitaram a divisão de tarefas pelos elementos do grupo, uma vez que permitiram detetar entidades em grande medida independentes. 

Na fase de testes de carga, encontraram-se diversas dificuldades devido à dificuldade de aprendizagem da ferramenta utilizada e problemas relacionados com concorrência. Adicionalmente, a inconstância dos resultados obtidos para configurações e testes equivalentes, dificultou o processo de avaliação de desempenho da aplicação desenvolvida. Salienta-se ainda a ausência de tempo de espera entre operações, que, apesar de ser irrealista, permite a realização de testes mais intensivos, que levam em consideração o pior caso.

Em relação aos testes de usabilidade na interface, verificou-se mais uma vez que a realização dos mesmo atempadamente, teria proporcionado a possibilidade de adaptar a interface, tornando-a mais intuitiva para o utilizador.

Em suma, este projeto providenciou ao grupo de trabalho os conhecimentos práticos necessários para o desenvolvimento de uma aplicação \textit{web}, focada no utilizador e na performance da mesma. Ainda que os resultados obtidos em termos de desempenho pudessem ser melhorados, o grupo adquiriu os conhecimentos necessários para avaliar a performance de um sistema \textit{web} em momentos de carga e efetuar as melhorias necessárias ao mesmo.
